% Created 2021-07-27 Ter 06:19
% Intended LaTeX compiler: pdflatex
\documentclass[11pt]{article}
\usepackage[utf8]{inputenc}
\usepackage[T1]{fontenc}
\usepackage{graphicx}
\usepackage{grffile}
\usepackage{longtable}
\usepackage{wrapfig}
\usepackage{rotating}
\usepackage[normalem]{ulem}
\usepackage{amsmath}
\usepackage{textcomp}
\usepackage{amssymb}
\usepackage{capt-of}
\usepackage{hyperref}
\usepackage[all]{xy}
\usepackage{proof,amsthm}
\newtheorem{theorem}{Theorem}[section]
\newtheorem{definition}{Definition}[section]
\newtheorem{lemma}{Lemma}[section]
\newcommand{\tto}{\twoheadrightarrow}
\author{Leandro O. Rezende, Flávio L. C. de Moura\thanks{flaviomoura@unb.br}}
\date{\today}
\title{A Formal Characterization of Confluence via Z Property}
\hypersetup{
 pdfauthor={Leandro O. Rezende, Flávio L. C. de Moura},
 pdftitle={A Formal Characterization of Confluence via Z Property},
 pdfkeywords={},
 pdfsubject={},
 pdfcreator={Emacs 26.1 (Org mode 9.2.2)}, 
 pdflang={English}}
\begin{document}

\maketitle

\section{Introduction}
\label{sec:org82831c6}

This work is about confluence, an essential property related to the
determinism of the computation in the sense that different reduction
paths all lead to the same result. In general, it is an undecidable
property that is difficult to be achieved.

A technique proposed by Dehornoy and van Oostrom, called the \emph{Z
property}, establishes a sufficient condition for the confluence of an
Abstract Rewriting System (ARS). This technique has been used for
proving confluence of several rewriting systems in a more direct way
including \(\lambda\)-calculi with and without explicit substitutions
\cite{kes09,2016arXiv160903139N}.

An ARS is a pair \((A,R)\) formed by a set \(A\) and a binary relation
\(R\) over this set. The decomposition of the binary relation \(R\)
into sub-relations allows \ldots{} For instance, in the
\(\lambda\)-calculus, the reduction \(\to_{\beta\eta}\) means
\(\to_{\beta} \cup \to_{\eta}\).

In this work, we

\section{Abstract Rewriting Systems}
\label{sec:org7a3a72a}

 When \((a,b) \in R\),
we write \(a R b\) or \(a \to_R b\) and we say that \(a\) \emph{reduces} to
\(b\) in one step. A finite reduction chain \(a \to_R a_1 \to_R \ldots
\to_R a_n\) means that \(a\) reduces to \(a_n\) in \(n\) steps, and we
write \(a \to_R^n a_n\). It is usual to write \(\to\) instead of
\(\to_R\), when \(R\) is clear. In this way, \(\to^+\)
(resp. \(\tto\)) denotes the transitive closure (resp. reflexive
transitive closure) of \(\to\). Several problems can be expressed in
terms of ARS (cf. \cite{BN98,DershowitzJ90,terese03}), and in this work
we study characterization of the confluence property. More precisely,
given an ARS \((A,\to)\), an element \(a\in A\) usually can be reduced
in different ways according to \(\to\). Informally, the confluence
property states that, no matter how the reduction is done, there is
always a way to join reduction paths. The following diagram is used to
express this idea:

\[
  \xymatrix{
    & a \ar@{->>}[dl] \ar@{->>}[dr] & \\
    b \ar@{.>>}[dr] &  & c \ar@{.>>}[dl] \\
    & d & 
  }
\]

The diagram states that if \(a\) can be reduced in two different ways
to \(b\) and \(c\), then there exists an element \(d\) in \(A\) such
that both \(b\) and \(c\) can be reduced to \(d\). 

The paper is organized as follows: In Section \ref{sec:org2a73302}, we present 

\section{Yet another proof that Z property implies confluence}
\label{sec:orgdea6d21}

In this section, we prove that Z property implies confluence for
ARSs. The novelty of this proof is that it is direct (for instance,
\cite{2016arXiv160903139N} establishes confluence from Z property via
semi-confluence), and does not rely on the law of excluded middle, or
other classical axioms
(cf. \cite{kesner09:_theor_explic_subst_with_safe_full_compos}). In
addition, it is formalized in the Coq proof assistant. The theorem is as follows:

\begin{theorem}
If the ARS \((A,\to)\) has the Z property then it is confluent.
\end{theorem}

The proof is as follows: Let \(a\in A\). Suppose that

\[
  \xymatrix{
    & a \ar@{->>}[dl] \ar@{->>}[dr] & \\
    b  &  & c 
  }
\]

\noindent for some \(b,c\in A\). We proceed by induction on \(a \tto
b\). If \(a=b\) the result is trivial, so suppose that \(a \to a' \tto
b\) for some \(a'\in A\). 

\[
  \xymatrix{
    & & a \ar@{->}[dl] \ar@{->>}[ddrr] & & \\
    & b' \ar@{->>}[dl] \ar@[blue]@{->>}[dr] & & & \\
    b \ar@[red]@{->>}[ddrr] & & x \ar@[red]@{->>}[dd] & &  c \ar@{.>>}[ddll] \\
    & & & & & \\
    & & d & &
   }
  \]

\[
  \xymatrix{
    & & a \ar@{->>}[dd] \ar@{->>}[ddrr] & & \\
    & b' \ar@{->>}[dl] \ar@{->>}[dr] & & & \\
    b \ar@{.>>}[ddrr] & & f(a) & &  c \ar@{.>>}[ddll] \\
    & & & & & \\
    & & d & &
   }
  \]

\[
      \xymatrix{
        & & a \ar@{->>}[dd] \ar@{->>}[ddrr] & & \\
        & b' \ar@{->>}[dl] \ar@{->>}[dr] & & & \\
        b \ar@{->>}[ddrr] & & f(a) \ar@{->>}[dd] & &  a \ar@{->>}[ll] \\
        & & & & & \\
        & & d & &
      }
    \]

\[
       \xymatrix{
        & & a \ar@{->>}[dd] \ar@{->}[dr] & & \\
        & b' \ar@{->>}[dl] \ar@{->>}[dr] \ar@[blue]@{->>}[ddr] & & c' \ar@{->>}[dr] \ar@[blue]@{->>}[ddl] & \\
        b \ar@[red]@{->>}[ddrr] & & f(a) & &  c  \ar@[red]@{->>}[ddll] \\
        & & f(c') & & & \\
        & & d  & & 
      }
     \]

\[
       \xymatrix{
        & & a \ar@{->>}[dd] \ar@{->}[dr] & & \\
        & b' \ar@{->>}[dl] \ar@{->>}[dr] & & c' \ar@{->>}[dr] \ar@{->>}[dl] & \\
        b \ar@{->>}[ddrr] \ar@{->>}[ddrr] & & f(a) \ar@{->>}[d] & &  c  \ar@{->>}[ddll] \\
        & & f(c') & & & \\
        & & d  & & 
      }
     \]
\section{The formalization}
\label{sec:org2a73302}

In \cite{ZPropertyDraft}, a sufficient condition for confluence is given
by the so called Z property:

\begin{definition}
Let \((A,\to)\) be an ARS. The system \((A,\to)\) has the Z property,
if there exists a map \(f:A \to A\) such that \(a \to b\) implies
that \(b \tto f(a)\) and \(f(a) \tto f(b)\):

\[
  \xymatrix{
   a \ar[r] &  b \ar@{.>>}[dl]\\
   f(a) \ar@{.>>}[r] & f(b) \\ 
  }
 \]
\end{definition}


\section{Conclusions and Future Work}
\label{sec:orgd7f8e05}

\bibliographystyle{alpha}
\bibliography{/Users/flaviomoura/Dropbox/bibliography/references} 
\end{document}