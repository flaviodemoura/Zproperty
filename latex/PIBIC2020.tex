\documentclass{entcs} 
\usepackage{prentcsmacro}
\usepackage{graphicx}
\usepackage{hyperref}
\usepackage[utf8]{inputenc}
\usepackage{mathpartir}
\usepackage[all]{xy}
\usepackage[english]{babel}
\usepackage[usenames]{color} 
\usepackage{amsmath}
\usepackage{amssymb}
\usepackage{pdfsync}
\usepackage{coqdoc}

\newcommand{\tto}{\twoheadrightarrow}

\def\lastname{de Moura {\it et. al.}}

\begin{document}
\begin{frontmatter}
  \title{A Confluência como Consequência da Propriedade Z em Coq}
  \author{Flávio L. C. de Moura\thanksref{flaviomoura}}
  \author{Leandro O. Rezende\thanksref{leandro}}
  \address{Departamento de Ciência da Computação, Universidade de Brasília, Brasília, Brasil}
  \thanks[flaviomoura]{Email:\href{flaviomoura@unb.br}{\texttt{\normalshape flaviomoura@unb.br}}} 
  \thanks[leandro]{Email:\href{l-ordo.ab.chao@hotmail.com}{\texttt{\normalshape l-ordo.ab.chao@hotmail.com}}}
  
\begin{abstract}
  A Teoria de Reescrita é um modelo computacional equivalente às
  máquinas de Turing, e seu sistema de reescrita de termos melhor conhecido 
  é o cálculo-$\lambda$. A Confluência é uma propriedade importante e indecidível 
  relacionada ao determinismo no processo computacional. Provas diretas 
  de confluência em sistemas de reescrita de termos são, no geral, difíceis de serem 
  feitas. Portanto, caracterizações alternativas da confluência podem 
  ajudar a contornar essa dificuldade em contextos distintos. Esse é o caso da propriedade 
  Z, que foi usada para provar confluência de diversos sistemas de reescrita, como 
  o cálculo-$\lambda$ com $\beta\eta$-redução, extensões do cálculo-$\lambda$ com
  substituições explícitas, o cálculo-$\lambda\mu$, etc. Apresentamos nesse artigo uma
  prova construtiva de que a propriedade Z implica na confluência. As provas conhecidas 
  desse fato geralmente dependem da lei do terceiro excluído, mas seu uso não é 
  necessário, como mostramos em nossa prova, que contorna essa necessidade usando 
  indução aninhada no passo indutivo da prova original.
  Ademais, nós formalizamos nossa prova, além de uma extensão da propriedade Z
  conhecida como propriedade Z composicional, no assistente de provas Coq.
\end{abstract}

\begin{keyword}
  Sistema de reescrita de termos, confluência, assistentes de provas, Coq
\end{keyword}
\end{frontmatter}

\section{Introdução}

A confluência é uma propriedade importante e indecidível acerca do 
determinismo de um processo computacional. Podemos dizer, nesse 
contexto, que um programa é confluente se quaisquer duas maneiras de 
avaliá-lo resultam na mesma resposta. No caso particular de sistemas de 
reescrita de termos (em inglês, "Abstract Rewriting Systems", ou ARS), que são o foco deste artigo, a confluência 
pode ser expressada de maneira elegante através do uso de diagramas, 
como veremos na próxima seção.

As contribuições deste artigo são as seguintes:
\begin{itemize}
\item Apresentamos uma prova costrutiva e baseada em indução aninhada 
de que a propriedade Z implica em confluência;
\item A prova de que a propriedade Z implica na confluência é formalizada no 
assistente de provas Coq, e sua apresentação é feita através da intercalação 
da explicação em português com o código correspondende em Coq. Assim, 
os comentários são feitos diretamente nos arquivos em Coq usando o 
estilo de comentário do coqdoc. Nós acreditamos que essa abordagem pode 
ser interessante para quem não seja familiarizado com o assistente de provas Coq, 
já que a intercalação de código com comentários em português dá uma boa ideia 
de como eles se relacionam.
\item Nós formalizamos uma extensão da propriedade Z, conhecida como propriedade 
Z composicional, descrita em \cite{Nakazawa-Fujita2016}.
\end{itemize}

\input{../src/ZtoConflPIBIC.tex}

% No diretório src:
% coqdoc --latex --body-only ZtoConfl.v

\section{Conclusão}

Nesse trabalho, apresentamos uma prova construtiva de que a 
propriedade Z implica na confluência, uma propriedade interessante 
para sistemas de reescrita de termos. Além disso, nós provamos 
esse resultado formalmente usando o assistente de provas Coq. 
Os arquivos contendo esses resultados estão disponíveis no nosso 
repositório do GitHub: \url{https://github.com/flaviodemoura/Zproperty}.

A propriedade Z foi apresentada inicialmente por V. van Oostrom como 
condição suficiente para um ARS ser confluente~\cite{zproperty}, e desde então foi usada 
para provar confluência em diversos contextos, como o cálculo-$\lambda$ com 
$\beta\eta$-redução, extensões do cálculo-$\lambda$ com substituições explícitas e 
o cálculo $\lambda\mu$. As provas em Coq dos resultados principais foram 
comentadas linha a linha, servindo tanto como uma apresentação informal das provas 
quanto como seu complemento formal. Ademais, nós formalizamos uma extensão da 
propriedade Z, conhecida como propriedade Z composicional, descrita em 
\cite{Nakazawa-Fujita2016}.

Futuramente, essa formalização será usada para provar a confluência de um 
cálculo com substituições explícitas baseado no cálculo-$\lambda_{ex}$ (cf.~\cite{kes09}). 
Por fim, esperamos que nossa formalização possa ser usada como um framework 
para prova de confluência em outros sistemas de reescrita.

\bibliographystyle{plain}
\bibliography{references.bib}

\end{document}
