\documentclass[a4paper]{article}
\usepackage{graphicx}
\usepackage{coqdoc}
\usepackage{amsmath}
\usepackage{amssymb}
\usepackage[all]{xy}
\usepackage{mathpartir}
\usepackage{tabto,calc}
\usepackage{onecolceurws}

\newcommand{\tto}{\twoheadrightarrow}
\newcommand{\flavio}[1]{{\color{red}#1}}

\newcommand{\comm}[1]{\tabto{\CurrentLineWidth}\fbox{\begin{minipage}[t]{0.98\linewidth-\TabPrevPos} #1 \end{minipage}}}

\newtheorem{theorem}{Theorem}[section]
\newtheorem{definition}{Definition}[section]
\newtheorem{lemma}{Lemma}[section]

\title{A Formalization of the (Compositional) Z Property}

\author{
Flávio L. C. de Moura and Leandro O. Rezende\\ Dept. de Ciência da Computação\\
                Universidade de Brasília \\ flaviomoura@unb.br
}

\institution{}




\begin{document}
\maketitle

\begin{abstract}
  Rewriting theory is a well established model of computation
  equivalent to the Turing machines, and the most well known rewriting
  system is the $\lambda$-calculus. Confluence is an important and
  undecidable property related to the determinism of the computational
  process. Direct proofs of confluence are, in general, difficult to
  be done. Therefore, alternative characterizations of confluence can
  circumvent this difficulty for different contexts. This is the case
  of the so called Z property, which has been successfully used to
  prove confluence in several situations such as the
  $\lambda$-calculus with $\beta\eta$-reduction, extensions of the
  $\lambda$-calculus with explicit substitutions, the
  $\lambda\mu$-calculus, etc. In this work we present a direct and
  constructive proof that the Z property implies confluence.  In
  addition, we formalized our proof and an extension of the Z
  property, known as the Compositional Z, in the Coq proof assistant.
\end{abstract}
\vskip 32pt


\section{Introduction}

Confluence is an important and undecidable property concerning the
determinism of the computational process. This means that
independently of the choice of the evaluation path, result is always
the same. In the particular case of Abstract Rewriting Systems (ARS),
which are the focus of this work, confluence can be beautifully
expressed by diagrams as we will see in the next section.

The contributions of this work are as follows:
\begin{itemize}
\item We present a \flavio{new(?)} proof that the Z property implies confluence,
  which is direct and constructive.
\item The proof that the Z property implies confluence is formalized
  in the Coq proof assistant,\flavio{and the presentation is made
  interleaving Coq code followed by an explanation in English of the
  code. In this way, the annotations are done directly in the Coq
  files using the coqdoc annotation style. We believe that this
  approach is interesting for those that are not familiar with the Coq
  proof assistant because the Coq code followed by English
  explanations gives a good idea on how they relate to each
  other. This discipline also forces a better organization of the
  formalization and of the proofs so that the explanation in English
  is comprehensible.}
\item We formalize an extension of the Z property, known as
  compositional Z property, as presented in
  \cite{Nakazawa-Fujita2016}.
\end{itemize}

\section{The Z property implies Confluence}

An ARS, say $(A,R)$, is a pair composed of a set $A$ and binary
relation over this set $R:A\times A$. Let $a,b\in A$. We write
$a\to_R b$ (or $R\ a\ b$ in Coq) to denote that $(a,b)\in R$, and in
this case, we say that $a$ $R$-reduces to $b$ in one step. The
reflexive transitive closure of a relation \coqdocvar{R}, written as
$\tto_R$, is defined by the following inference rules: \begin{mathpar}
  \inferrule*[Right={$(refl)$}]{~}{a \tto_R a} \and
  \inferrule*[Right={$(rtrans)$}]{a\to_R b \and b \tto_R c}{a \tto_R
    c} \end{mathpar} \noindent where $a,b$ and $c$ are universally
quantified variables as explicitly stated in the corresponding Coq
definition: \begin{coqdoccode} \coqdocemptyline \coqdocnoindent
  \coqdockw{Inductive} \coqdocvar{refltrans}
  \{\coqdocvar{A}:\coqdockw{Type}\} (\coqdocvar{R}: \coqdocvar{Rel}
  \coqdocvar{A}) : \coqdocvar{A} \ensuremath{\rightarrow}
  \coqdocvar{A} \ensuremath{\rightarrow} \coqdockw{Prop} :=\coqdoceol
  \coqdocnoindent \ensuremath{|} \coqdocvar{refl}:
  \coqdockw{\ensuremath{\forall}} \coqdocvar{a},
  (\coqdocvar{refltrans} \coqdocvar{R}) \coqdocvar{a}
  \coqdocvar{a}\coqdoceol \coqdocnoindent \ensuremath{|}
  \coqdocvar{rtrans}: \coqdockw{\ensuremath{\forall}} \coqdocvar{a}
  \coqdocvar{b} \coqdocvar{c}, \coqdocvar{R} \coqdocvar{a}
  \coqdocvar{b} \ensuremath{\rightarrow} \coqdocvar{refltrans}
  \coqdocvar{R} \coqdocvar{b} \coqdocvar{c} \ensuremath{\rightarrow}
  \coqdocvar{refltrans} \coqdocvar{R} \coqdocvar{a}
  \coqdocvar{c}.\coqdoceol \coqdocemptyline
\end{coqdoccode}

The reflexive transitive closure of a relation is used to define
    the notion of confluence: no matter how the reduction is done, the
    result will always be the same. In other words, every divergence
    is joinable as stated by the following diagram:


    $\centerline{\xymatrix{ & a \ar@{->>}[dl] \ar@{->>}[dr] & \\ b
    \ar@{.>>}[dr] & & c \ar@{.>>}[dl] \\ & d & }}$


Formally, this means that if an expression $a$ can be reduced in two
different ways to $b$ and $c$, then there exists an expression $d$
such that both $b$ and $c$ reduce to $d$. The existential
quantification is expressed by the dotted lines in the diagram. This
notion is defined in the Coq system as follows: \begin{coqdoccode}
  \coqdocemptyline \coqdocnoindent \coqdockw{Definition}
  \coqdocvar{Confl} \{\coqdocvar{A}:\coqdockw{Type}\} (\coqdocvar{R}:
  \coqdocvar{Rel} \coqdocvar{A}) := \coqdockw{\ensuremath{\forall}}
  \coqdocvar{a} \coqdocvar{b} \coqdocvar{c}, (\coqdocvar{refltrans}
  \coqdocvar{R}) \coqdocvar{a} \coqdocvar{b} \ensuremath{\rightarrow}
  (\coqdocvar{refltrans} \coqdocvar{R}) \coqdocvar{a} \coqdocvar{c}
  \ensuremath{\rightarrow} (\coqdoctac{\ensuremath{\exists}}
  \coqdocvar{d}, (\coqdocvar{refltrans} \coqdocvar{R}) \coqdocvar{b}
  \coqdocvar{d} \ensuremath{\land} (\coqdocvar{refltrans}
  \coqdocvar{R}) \coqdocvar{c} \coqdocvar{d}).\coqdoceol
  \coqdocemptyline
\end{coqdoccode}

In \cite{dehornoy2008z}, V. van Oostrom gives a sufficient condition
for an ARS to be confluent. This condition is based on the \textit{Z
  Property} that is defined as follows:


\begin{definition} Let $(A,\to_R)$ be an ARS. A mapping $f:A \to A$ 
  satisfies the Z property for $\to_R$, if $a \to_R b$ implies
  $b \tto_R f a  \tto_R f b$, for any $a, b \in A$. 
\end{definition}

The name of the property comes from the following diagrammatic
representation of this definition:
    
\[ \xymatrix{ a \ar[r]_R & b \ar@{.>>}[dl]^R\\ f a \ar@{.>>}[r]_R & f
    b \\ } \]

If a function \coqdocvar{f} satisfies the Z property for $\to_R$ then
we say that \coqdocvar{f} is Z for $\to_R$, and the corresponding Coq
definition is given by the following predicate:

\begin{coqdoccode}
  \coqdocemptyline \coqdocnoindent \coqdockw{Definition}
  \coqdocvar{f\_is\_Z} \{\coqdocvar{A}:\coqdockw{Type}\}
  (\coqdocvar{R}: \coqdocvar{Rel} \coqdocvar{A}) (\coqdocvar{f}:
  \coqdocvar{A} \ensuremath{\rightarrow} \coqdocvar{A}) :=
  \coqdockw{\ensuremath{\forall}} \coqdocvar{a} \coqdocvar{b},
  \coqdocvar{R} \coqdocvar{a} \coqdocvar{b} \ensuremath{\rightarrow}
  ((\coqdocvar{refltrans} \coqdocvar{R}) \coqdocvar{b} (\coqdocvar{f}
  \coqdocvar{a}) \ensuremath{\land} (\coqdocvar{refltrans}
  \coqdocvar{R}) (\coqdocvar{f} \coqdocvar{a}) (\coqdocvar{f}
  \coqdocvar{b})).\coqdoceol \coqdocemptyline
\end{coqdoccode}

Alternatively, an ARS $(A,\to_R)$ satisfies the Z property if there
exists a mapping $f:A \to A$ such that $f$ is Z for $\to_R$:

\begin{coqdoccode}
  \coqdocemptyline \coqdocnoindent \coqdockw{Definition}
  \coqdocvar{Z\_prop} \{\coqdocvar{A}:\coqdockw{Type}\}
  (\coqdocvar{R}: \coqdocvar{Rel} \coqdocvar{A}) :=
  \coqdoctac{\ensuremath{\exists}} \coqdocvar{f}:\coqdocvar{A}
  \ensuremath{\rightarrow} \coqdocvar{A},
  \coqdockw{\ensuremath{\forall}} \coqdocvar{a} \coqdocvar{b},
  \coqdocvar{R} \coqdocvar{a} \coqdocvar{b} \ensuremath{\rightarrow}
  ((\coqdocvar{refltrans} \coqdocvar{R}) \coqdocvar{b} (\coqdocvar{f}
  \coqdocvar{a}) \ensuremath{\land} (\coqdocvar{refltrans}
  \coqdocvar{R}) (\coqdocvar{f} \coqdocvar{a}) (\coqdocvar{f}
  \coqdocvar{b})).\coqdoceol \coqdocemptyline
\end{coqdoccode}

The first contribution of this work is a constructive proof of the
fact that the Z property implies confluence. Our proof uses nested
induction, and hence it differs from the one in \cite{kes09} (that
follows \cite{dehornoy2008z}) and the one in \cite{zproperty} in the
sense that it does not rely on the analyses of whether a term is in
normal form or not, avoiding the necessity of the law of the excluded
middle . As a result, we have an elegant inductive proof of the fact
that if an ARS satisfies the Z property then it is confluent. This
proof is formalized in the Coq proof assistant, and the whole
formalization is available in a GitHub
repository\footnote{\url{https://github.com/flaviodemoura/Zproperty}}.

In what follows, we present the theorem and its proof interleaving Coq
code and the corresponding comments.
\begin{coqdoccode}
  \coqdocemptyline \coqdocnoindent \coqdockw{Theorem}
  \coqdocvar{Z\_prop\_implies\_Confl}
  \{\coqdocvar{A}:\coqdockw{Type}\}: \coqdockw{\ensuremath{\forall}}
  \coqdocvar{R}: \coqdocvar{Rel} \coqdocvar{A}, \coqdocvar{Z\_prop}
  \coqdocvar{R} \ensuremath{\rightarrow} \coqdocvar{Confl}
  \coqdocvar{R}.
\end{coqdoccode}

\flavio{Add a short (informal) proof}
  
An alternative proof that Z implies confluence is possible via the
notion of semiconfluence, which is equivalent to confluence, as done
in \cite{zproperty}. Unlike the proof in \cite{zproperty} and
similarly to our previous proof, our proof of the Theorem that Z
implies semiconfluence is constructive, but we will not explain it
here due to lack of space; an interested reader can find it in the
source code file.

\section{An extension of the Z property: Compositional Z}

In this section we present a formalization of an extension of the Z
property with compositional functions, known as \textit{Compositional
  Z}, as presented in \cite{Nakazawa-Fujita2016}. The compositional Z
is an interesting property because it allows a kind of modular
approach to the Z property in such a way that the reduction relation
can be split into two parts. More precisely, given an ARS $(A,\to_R)$,
one must be able to decompose the relation $\to_R$ into two parts, say
$\to_1$ and $\to_2$ such that $\to_R = \to_1\cup \to_2$. This kind of
decomposition can be done in several interesting situations such as
the $\lambda$-calculus with $\beta\eta$-reduction\cite{Ba84},
extensions of the $\lambda$-calculus with explicit
substitutions\cite{accl91}, the $\lambda\mu$-calculus\cite{Parigot92},
etc. But before presenting the full definition of the Compositional Z,
we need to define the \textit{weak Z property}:


\begin{figure}[h] \centering \[ \xymatrix{ a \ar[r]_R & b
      \ar@{.>>}[dl]^x\\ f(a) \ar@{.>>}[r]_x & f(b) \\ } \]
  \caption{The weak Z property}\label{fig:weakZ}
\end{figure}


\begin{definition} Let $(A,\to_R)$ be an ARS and $\to_R'$ a
  relation on $A$. A mapping $f$ satisfies the {\it weak Z
    property} for $\to_R$ by $\to_R'$ if $a\to_R b$ implies $b \tto_R'
  f(a)$ and $f(a) \tto_R' f(b)$ (cf. Figure
  \ref{fig:weakZ}). Therefore, a mapping $f$ satisfies the Z
  property for $\to_R$ if it satisfies the weak Z property by
  itself.
\end{definition}


When $f$ satisfies the weak Z property, we also say that $f$ is weakly
Z, and the corresponding definition in Coq is given as
follows: \begin{coqdoccode} \coqdocemptyline \coqdocnoindent
  \coqdockw{Definition} \coqdocvar{f\_is\_weak\_Z} \{\coqdocvar{A}\}
  (\coqdocvar{R} \coqdocvar{R'}: \coqdocvar{Rel} \coqdocvar{A})
  (\coqdocvar{f}: \coqdocvar{A} \ensuremath{\rightarrow}
  \coqdocvar{A}) := \coqdockw{\ensuremath{\forall}} \coqdocvar{a}
  \coqdocvar{b}, \coqdocvar{R} \coqdocvar{a} \coqdocvar{b}
  \ensuremath{\rightarrow} ((\coqdocvar{refltrans} \coqdocvar{R'})
  \coqdocvar{b} (\coqdocvar{f} \coqdocvar{a}) \ensuremath{\land}
  (\coqdocvar{refltrans} \coqdocvar{R'}) (\coqdocvar{f} \coqdocvar{a})
  (\coqdocvar{f} \coqdocvar{b})).\coqdoceol \coqdocemptyline
\end{coqdoccode}

The compositional Z is an extension of the Z property for
compositional functions, where composition is defined as
usual:

\begin{coqdoccode} \coqdocemptyline \coqdocnoindent
  \coqdockw{Definition} \coqdocvar{comp} \{\coqdocvar{A}\}
  (\coqdocvar{f1} \coqdocvar{f2}: \coqdocvar{A}
  \ensuremath{\rightarrow} \coqdocvar{A}) := \coqdockw{fun}
  \coqdocvar{x}:\coqdocvar{A} \ensuremath{\Rightarrow} \coqdocvar{f1}
  (\coqdocvar{f2} \coqdocvar{x}).\coqdoceol \coqdocnoindent
  \coqdockw{Notation} "f1 \# f2" := (\coqdocvar{comp} \coqdocvar{f1}
  \coqdocvar{f2}) (\coqdoctac{at} \coqdockw{level} 40).\coqdoceol
  \coqdocemptyline
\end{coqdoccode}

\noindent and the disjoint union is inductively defined as:

\begin{coqdoccode}
  \coqdocemptyline \coqdocnoindent \coqdockw{Inductive}
  \coqdocvar{union} \{\coqdocvar{A}\} (\coqdocvar{red1}
  \coqdocvar{red2}: \coqdocvar{Rel} \coqdocvar{A}) : \coqdocvar{Rel}
  \coqdocvar{A} :=\coqdoceol \coqdocnoindent \ensuremath{|}
  \coqdocvar{union\_left}: \coqdockw{\ensuremath{\forall}}
  \coqdocvar{a} \coqdocvar{b}, \coqdocvar{red1} \coqdocvar{a}
  \coqdocvar{b} \ensuremath{\rightarrow} \coqdocvar{union}
  \coqdocvar{red1} \coqdocvar{red2} \coqdocvar{a}
  \coqdocvar{b}\coqdoceol \coqdocnoindent \ensuremath{|}
  \coqdocvar{union\_right}: \coqdockw{\ensuremath{\forall}}
  \coqdocvar{a} \coqdocvar{b}, \coqdocvar{red2} \coqdocvar{a}
  \coqdocvar{b} \ensuremath{\rightarrow} \coqdocvar{union}
  \coqdocvar{red1} \coqdocvar{red2} \coqdocvar{a}
  \coqdocvar{b}.\coqdoceol \coqdocnoindent \coqdockw{Notation} "R1
  !\_! R2" := (\coqdocvar{union} \coqdocvar{R1} \coqdocvar{R2})
  (\coqdoctac{at} \coqdockw{level} 40).\coqdoceol \coqdocemptyline
  \coqdocemptyline
\end{coqdoccode}

We are now ready to present the definition of the compositional Z:

\begin{theorem}\cite{Nakazawa-Fujita2016}\label{thm:zcomp} Let
  $(A,\to_R)$ be an ARS such that $\to_R = \to_1 \cup \to_2$. If there
  exists mappings $f_1,f_2: A \to A$ such that
  \begin{enumerate} \item $f_1$ is Z for $\to_1$ \item $a \to_1 b$
    implies $f_2(a) \tto f_2(b)$ \item $a \tto f_2(a)$ holds for any
    $a\in Im(f_1)$ \item $f_2 \circ f_1$ is weakly Z for $\to_2$ by
    $\to_R$
  \end{enumerate} then $f_2 \circ f_1$ is Z for
  $(A,\to_R)$, and hence $(A,\to_R)$ is confluent.
\end{theorem}


We define the predicate \coqdocvar{Z\_comp} that corresponds to the
premises of Theorem \ref{thm:zcomp}, i.e. to the conjunction of items
(i), (ii), (iii) and (iv) in addition to the fact that
$\to_R = \to_1 \cup \to_2$, where $\to_1$ (resp. $\to_2$) is written
as \coqdocvar{R1} (resp. \coqdocvar{R2}):

\begin{coqdoccode}
  \coqdocemptyline \coqdocnoindent \coqdockw{Definition}
  \coqdocvar{Z\_comp} \{\coqdocvar{A}:\coqdockw{Type}\} (\coqdocvar{R}
  :\coqdocvar{Rel} \coqdocvar{A}) := \coqdoctac{\ensuremath{\exists}}
  (\coqdocvar{R1} \coqdocvar{R2}: \coqdocvar{Rel} \coqdocvar{A})
  (\coqdocvar{f1} \coqdocvar{f2}: \coqdocvar{A}
  \ensuremath{\rightarrow} \coqdocvar{A}), \coqdocvar{R} =
  (\coqdocvar{R1} !\coqdocvar{\_}! \coqdocvar{R2}) \ensuremath{\land}
  \coqdocvar{f\_is\_Z} \coqdocvar{R1} \coqdocvar{f1}
  \ensuremath{\land} (\coqdockw{\ensuremath{\forall}} \coqdocvar{a}
  \coqdocvar{b}, \coqdocvar{R1} \coqdocvar{a} \coqdocvar{b}
  \ensuremath{\rightarrow} (\coqdocvar{refltrans} \coqdocvar{R})
  (\coqdocvar{f2} \coqdocvar{a}) (\coqdocvar{f2} \coqdocvar{b}))
  \ensuremath{\land} (\coqdockw{\ensuremath{\forall}} \coqdocvar{a}
  \coqdocvar{b}, \coqdocvar{b} = \coqdocvar{f1} \coqdocvar{a}
  \ensuremath{\rightarrow} (\coqdocvar{refltrans} \coqdocvar{R})
  \coqdocvar{b} (\coqdocvar{f2} \coqdocvar{b})) \ensuremath{\land}
  (\coqdocvar{f\_is\_weak\_Z} \coqdocvar{R2} \coqdocvar{R}
  (\coqdocvar{f2} \# \coqdocvar{f1})).\coqdoceol \coqdocemptyline
  \coqdocemptyline
\end{coqdoccode}

As stated by Theorem \ref{thm:zcomp}, the compositional Z gives
    a sufficient condition for compositional functions to be Z. In
    other words, compositional Z implies Z, which is justified by the
    diagrams of Figure \ref{fig:zcomp}.


    \begin{figure}[h]\begin{tabular}{l@{\hskip 3cm}l} $\xymatrix{ a
    \ar@{->}[rr]^1 && b \ar@{.>>}[dll]_1\\ f_1(a)\ar@{.>>}[d]
    \ar@{.>>}[rr]^1 && f_1(b) \\ f_2(f_1(a)) \ar@{.>>}[rr] &&
    f_2(f_1(b)) }$ & $\xymatrix{ a \ar@{->}[rr]^2 && b
    \ar@{.>>}[ddll]\\ & & \\ f_2(f_1(a)) \ar@{.>>}[rr] && f_2(f_1(b))
    }$ \end{tabular}\caption{Compositional Z implies
    Z}\label{fig:zcomp}\end{figure}


    In what follows, we present our commented Coq proof of this fact:
    \begin{coqdoccode}
\coqdocemptyline
\coqdocnoindent
\coqdockw{Theorem} \coqdocvar{Z\_comp\_implies\_Z\_prop} \{\coqdocvar{A}:\coqdockw{Type}\}: \coqdockw{\ensuremath{\forall}} (\coqdocvar{R} :\coqdocvar{Rel} \coqdocvar{A}), \coqdocvar{Z\_comp} \coqdocvar{R} \ensuremath{\rightarrow} \coqdocvar{Z\_prop} \coqdocvar{R}.\coqdoceol
\coqdocnoindent
\coqdockw{Proof}.\coqdoceol
\coqdocindent{1.00em}
\coqdoctac{intros} \coqdocvar{R} \coqdocvar{H}. \end{coqdoccode}
\comm{Let $R$ be a relation over $A$, and $H$ the
      hypothesis that $R$ satisfies the compositional Z.} \begin{coqdoccode}
\coqdocemptyline
\coqdocindent{1.00em}
\coqdoctac{unfold} \coqdocvar{Z\_prop}. \coqdoctac{unfold} \coqdocvar{Z\_comp} \coqdoctac{in} \coqdocvar{H}. \coqdoctac{destruct} \coqdocvar{H} \coqdockw{as}\coqdoceol
\coqdocindent{1.00em}
[ \coqdocvar{R1} [ \coqdocvar{R2} [\coqdocvar{f1} [\coqdocvar{f2} [\coqdocvar{Hunion} [\coqdocvar{H1} [\coqdocvar{H2} [\coqdocvar{H3} \coqdocvar{H4}]]]]]]]]. \end{coqdoccode}
\comm{Now
      unfold the definitions of $Z\_prop$ and $Z\_comp$ as presented
      before, and name the hypothesis of the compositional Z as in
      Theorem \ref{thm:zcomp}. We need to prove that there exists a
      map, say $f$, that is Z as shown by the current proof context:
      \newline

      \includegraphics[scale=0.4]{figs/fig8.png} } \begin{coqdoccode}
\coqdocemptyline
\coqdocindent{1.00em}
\coqdoctac{\ensuremath{\exists}} (\coqdocvar{f2} \# \coqdocvar{f1}). \end{coqdoccode}
\comm{We will prove that the composition $f_2 \circ f_1$
  is Z.} \begin{coqdoccode}
\coqdocemptyline
\coqdocindent{1.00em}
\coqdoctac{intros} \coqdocvar{a} \coqdocvar{b} \coqdocvar{HR}. \end{coqdoccode}
\comm{Let $a$ and $b$ be elements of $A$, and suppose
  that $a$ $R$-reduces to $b$ in one step, i.e. that $a \to_R b$ and
  call $HR$ this hypothesis.} \begin{coqdoccode}
\coqdocemptyline
\coqdocindent{1.00em}
\coqdoctac{inversion} \coqdocvar{Hunion}; \coqdoctac{subst}. \coqdoctac{clear} \coqdocvar{H}. \coqdoctac{inversion} \coqdocvar{HR}; \coqdoctac{subst}. \end{coqdoccode}
\comm{Since
  $R$ is the union of $R1$ and $R2$, one has that $a$ reduces to $b$
  in one step via either $R1$ or $R2$. Therefore, there are two cases
  to consider:} \begin{coqdoccode}
\coqdocemptyline
\coqdocindent{1.00em}
- \coqdoctac{split}. \end{coqdoccode}
\comm{Firstly, suppose that $a$ $R1$-reduces in one step to
    $b$, i.e. $a \to_{R1} b$.} \begin{coqdoccode}
\coqdocemptyline
\coqdocindent{2.00em}
+ \coqdoctac{apply} \coqdocvar{refltrans\_composition} \coqdockw{with} (\coqdocvar{f1} \coqdocvar{a}). \end{coqdoccode}
\comm{In order to prove
    that $b \tto_R (f_2 (f_1\ a))$, we first need to show that $b
    \tto_{R1} (f_1\ a)$, and then that $(f_1\ a) \tto_R (f_2 (f_1\ a))$ as
    shown in Figure \ref{fig:zcomp}.} \begin{coqdoccode}
\coqdocemptyline
\coqdocindent{3.00em}
\ensuremath{\times} \coqdoctac{apply} \coqdocvar{H1} \coqdoctac{in} \coqdocvar{H}. \coqdoctac{destruct} \coqdocvar{H}. \coqdoctac{apply} \coqdocvar{refltrans\_union}; \coqdoctac{assumption}. \end{coqdoccode}
    \comm{The proof of $b \tto_{R1} (f_1\ a)$ is done from the fact that $f_1$
    is Z for $R1$.} \begin{coqdoccode}
\coqdocemptyline
\coqdocindent{3.00em}
\ensuremath{\times} \coqdoctac{apply} \coqdocvar{H3} \coqdockw{with} \coqdocvar{a}; \coqdoctac{reflexivity}. \end{coqdoccode}
\comm{The proof that $(f_1\ a)
    \tto_R (f_2 (f_1\ a))$ is a direct consequence of the hypothesis
    $H3$.} \begin{coqdoccode}
\coqdocemptyline
\coqdocindent{2.00em}
+ \coqdoctac{apply} \coqdocvar{H1} \coqdoctac{in} \coqdocvar{H}. \coqdoctac{destruct} \coqdocvar{H}. \coqdoctac{clear} \coqdocvar{H} \coqdocvar{HR}. \coqdoctac{unfold} \coqdocvar{comp}. \end{coqdoccode}
\comm{The
    proof that $(f_2 (f_1\ a))$ $R$-reduces to $(f_2 (f_1\ b))$ is
    more tricky. Initially, note that, since $a \to_{R1} b$ then we
    get that $(f_1\ a) \tto_{R1} (f_1\ b)$ by the Z property.} \begin{coqdoccode}
\coqdocemptyline
\coqdocindent{3.00em}
\coqdoctac{induction} \coqdocvar{H0}. \end{coqdoccode}
\comm{Now, the goal can be obtained from $H2$ as
      long as $(f_1\ a) \to_{R1} (f_1\ b)$, but from the hypothesis
      $H0$ we have that $(f_1\ a) \tto_{R1} (f_1\ b)$. Therefore, we
      proceed by induction on $H0$.} \begin{coqdoccode}
\coqdocemptyline
\coqdocindent{3.00em}
\ensuremath{\times} \coqdoctac{apply} \coqdocvar{refl}. \end{coqdoccode}
\comm{The reflexive case is trivial because $a$ and
        $b$ are equal.} \begin{coqdoccode}
\coqdocemptyline
\coqdocindent{3.00em}
\ensuremath{\times} \coqdoctac{apply} \coqdocvar{refltrans\_composition}\coqdoceol
\coqdocindent{7.00em}
\coqdoceol
\coqdocindent{4.00em}
\coqdockw{with} (\coqdocvar{f2} \coqdocvar{b0}). \end{coqdoccode}
\comm{In the transitive case, we have that $(f_1\ a)$ $R1$-reduces to
        $(f_1\ b)$ in at least one step. The current proof context is
        as follows, up to renaming of variables:

        \includegraphics[scale=0.5]{figs/fig9.png}

      Therefore, there exists some element $b0$ such that $a0\to_{R1}
      b0$ and $b0 \tto_{R1} c$ and we need to prove that $(f_2\ a0)
      \tto_{R1\cup R2} (f_2\ c)$. This can be done in two steps using
      the transitivity of [refltrans] taking $(f_2\ b0)$ as the
      intermediary term.} \begin{coqdoccode}
\coqdocemptyline
\coqdocindent{4.00em}
** \coqdoctac{apply} \coqdocvar{H2}; \coqdoctac{assumption}. \end{coqdoccode}
\comm{The first subgoal is then $(f_2\
           a0)\tto_{(R1 \cup R2)} (f_2\ b0)$ that is proved by
           hypothesis $H2$.} \begin{coqdoccode}
\coqdocemptyline
\coqdocindent{4.00em}
** \coqdoctac{assumption}. \end{coqdoccode}
\comm{And the second subgoal $(f_2\ b0) \tto_{(R1
           \cup R2)} (f_2\ c)$ is proved by the induction
           hypothesis.} \begin{coqdoccode}
\coqdocemptyline
\coqdocindent{1.00em}
- \coqdoctac{apply} \coqdocvar{H4}; \coqdoctac{assumption}. \end{coqdoccode}
\comm{Finally, when $a$ $R2$-reduces in one
    step to $b$ one concludes the proof using the assumption that
    $(f_2 \circ f_1)$ is weak Z.} \begin{coqdoccode}
\coqdocemptyline
\coqdocnoindent
\coqdockw{Qed}.\coqdoceol
\coqdocemptyline
\end{coqdoccode}
Now we can use the proofs of the theorems \coqdocvar{Z\_comp\_implies\_Z\_prop}
and \coqdocvar{Z\_prop\_implies\_Confl} to conclude that compositional Z is a
sufficient condition for confluence. \begin{coqdoccode}
\coqdocemptyline
\coqdocnoindent
\coqdockw{Corollary} \coqdocvar{Z\_comp\_is\_Confl} \{\coqdocvar{A}\}: \coqdockw{\ensuremath{\forall}} (\coqdocvar{R}: \coqdocvar{Rel} \coqdocvar{A}), \coqdocvar{Z\_comp} \coqdocvar{R} \ensuremath{\rightarrow} \coqdocvar{Confl} \coqdocvar{R}.\coqdoceol
\coqdocnoindent
\coqdockw{Proof}.\coqdoceol
\coqdocindent{1.00em}
\coqdoctac{intros} \coqdocvar{R} \coqdocvar{H}.\coqdoceol
\coqdocindent{1.00em}
\coqdoctac{apply} \coqdocvar{Z\_comp\_implies\_Z\_prop} \coqdoctac{in} \coqdocvar{H}.\coqdoceol
\coqdocindent{1.00em}
\coqdoctac{apply} \coqdocvar{Z\_prop\_implies\_Confl}; \coqdoctac{assumption}.\coqdoceol
\coqdocnoindent
\coqdockw{Qed}.\coqdoceol
\coqdocemptyline
\end{coqdoccode}
Rewriting Systems with equations is another interesting and
    non-trivial topic \cite{winkler89,terese03}. The confluence of
    rewriting systems with an equivalence relation can also be proved
    by a variant of the compositional Z, known as Z property
    modulo~\cite{AK12b}.


    \begin{theorem}\label{cor:zcomp} Let
     $(A,\to_R)$ be an ARS such that $\to_R = \to_1 \cup \to_2$. If
     there exist mappings $f_1,f_2: A \to A$ such that
     \begin{enumerate} \item $a \to_1 b$ implies $f_1(a) = f_1(b)$
     \item $a \tto_1 f_1(a), for all a$ \item $a \tto_R f_2(a)$ holds
     for any $a\in Im(f_1)$ \item $f_2 \circ f_1$ is weakly Z for
     $\to_2$ by $\to_R$ \end{enumerate} then $f_2 \circ f_1$ is Z for
     $(A,\to_R)$, and hence $(A,\to_R)$ is confluent. \end{theorem}


    We define the predicate \coqdocvar{Z\_comp\_eq} corresponding to the
    hypothesis of Theorem \ref{cor:zcomp}, and then we prove
    directly that if \coqdocvar{Z\_comp\_eq} holds for a relation \coqdocvar{R} then \coqdocvar{Zprop}
    \coqdocvar{R} also holds. This approach differs from
    \cite{Nakazawa-Fujita2016} that proves Theorem
    \ref{cor:zcomp}, which is a Corollary in \cite{Nakazawa-Fujita2016}, 
    directly from Theorem \ref{thm:zcomp} \begin{coqdoccode}
\coqdocemptyline
\coqdocnoindent
\coqdockw{Definition} \coqdocvar{Z\_comp\_eq} \{\coqdocvar{A}:\coqdockw{Type}\} (\coqdocvar{R} :\coqdocvar{Rel} \coqdocvar{A}) := \coqdoctac{\ensuremath{\exists}} (\coqdocvar{R1} \coqdocvar{R2}: \coqdocvar{Rel} \coqdocvar{A}) (\coqdocvar{f1} \coqdocvar{f2}: \coqdocvar{A} \ensuremath{\rightarrow} \coqdocvar{A}), \coqdocvar{R} = (\coqdocvar{R1} !\coqdocvar{\_}! \coqdocvar{R2}) \ensuremath{\land} (\coqdockw{\ensuremath{\forall}} \coqdocvar{a} \coqdocvar{b}, \coqdocvar{R1} \coqdocvar{a} \coqdocvar{b} \ensuremath{\rightarrow} (\coqdocvar{f1} \coqdocvar{a}) = (\coqdocvar{f1} \coqdocvar{b})) \ensuremath{\land} (\coqdockw{\ensuremath{\forall}} \coqdocvar{a}, (\coqdocvar{refltrans} \coqdocvar{R1}) \coqdocvar{a} (\coqdocvar{f1} \coqdocvar{a})) \ensuremath{\land} (\coqdockw{\ensuremath{\forall}} \coqdocvar{b} \coqdocvar{a}, \coqdocvar{a} = \coqdocvar{f1} \coqdocvar{b} \ensuremath{\rightarrow} (\coqdocvar{refltrans} \coqdocvar{R}) \coqdocvar{a} (\coqdocvar{f2} \coqdocvar{a})) \ensuremath{\land} (\coqdocvar{f\_is\_weak\_Z} \coqdocvar{R2} \coqdocvar{R} (\coqdocvar{f2} \# \coqdocvar{f1})).\coqdoceol
\coqdocemptyline
\coqdocnoindent
\coqdockw{Lemma} \coqdocvar{Z\_comp\_eq\_implies\_Z\_prop} \{\coqdocvar{A}:\coqdockw{Type}\}: \coqdockw{\ensuremath{\forall}} (\coqdocvar{R} : \coqdocvar{Rel} \coqdocvar{A}), \coqdocvar{Z\_comp\_eq} \coqdocvar{R} \ensuremath{\rightarrow} \coqdocvar{Z\_prop} \coqdocvar{R}.\coqdoceol
\coqdocnoindent
\coqdockw{Proof}.\coqdoceol
\coqdocindent{1.00em}
\coqdoctac{intros} \coqdocvar{R} \coqdocvar{Heq}. \coqdoctac{unfold} \coqdocvar{Z\_comp\_eq} \coqdoctac{in} \coqdocvar{Heq}. \end{coqdoccode}
\comm{Let $R$ be a relation
  and suppose that $R$ satisfies the predicate $Z\_comp\_eq$.} \begin{coqdoccode}
\coqdocemptyline
\coqdocindent{1.00em}
\coqdoctac{destruct} \coqdocvar{Heq} \coqdockw{as} [\coqdocvar{R1} [\coqdocvar{R2} [\coqdocvar{f1} [\coqdocvar{f2} [\coqdocvar{Hunion} [\coqdocvar{H1} [\coqdocvar{H2} [\coqdocvar{H3} \coqdocvar{H4}]]]]]]]]. \end{coqdoccode}
  \comm{Call $Hi$ the $i$th hypothesis as in \ref{cor:zcomp}.} \begin{coqdoccode}
\coqdocemptyline
\coqdocindent{1.00em}
\coqdoctac{unfold} \coqdocvar{Z\_prop}. \coqdoctac{\ensuremath{\exists}} (\coqdocvar{f2} \# \coqdocvar{f1}). \end{coqdoccode}
\comm{From the definition of the
  predicate $Z\_prop$, we need to find a map, say $f$ that is Z. Let
  $(f_2 \circ f_1)$ be such map.}  \begin{coqdoccode}
\coqdocemptyline
\coqdocindent{1.00em}
\coqdoctac{intros} \coqdocvar{a} \coqdocvar{b} \coqdocvar{Hab}. \end{coqdoccode}
\comm{In order to prove that $(f_2 \circ f_1)$ is Z,
  let $a$ and $b$ be arbitrary elements of type $A$, and $Hab$ be the
  hypothesis that $a \to_{R} b$.} \begin{coqdoccode}
\coqdocemptyline
\coqdocindent{1.00em}
\coqdoctac{inversion} \coqdocvar{Hunion}; \coqdoctac{subst}; \coqdoctac{clear} \coqdocvar{H}. \coqdoctac{inversion} \coqdocvar{Hab}; \coqdoctac{subst}; \coqdoctac{clear} \coqdocvar{Hab}. \end{coqdoccode}
  \comm{Since $a$ $R$-reduces in one step to $b$ and $R$ is the union of the
  relations $R1$ and $R2$ then we consider two cases:} \begin{coqdoccode}
\coqdocemptyline
\coqdocindent{1.00em}
- \coqdoctac{unfold} \coqdocvar{comp}; \coqdoctac{split}. \end{coqdoccode}
\comm{The first case is when $a \to_{R1}
    b$. This is equivalent to say that $f_2 \circ f_1$ is weak Z for
    $R1$ by $R1 \cup R2$.} \begin{coqdoccode}
\coqdocemptyline
\coqdocindent{2.00em}
+ \coqdoctac{apply} \coqdocvar{refltrans\_composition} \coqdockw{with} (\coqdocvar{f1} \coqdocvar{b}). \end{coqdoccode}
\comm{Therefore, we first
    prove that $b \tto_{(R1\cup R2)} (f_2 (f_1\ a))$, which can be
    reduced to $b \tto_{(R1\cup R2)} (f_1\ b)$ and $(f_1\ b)
    \tto_{(R1\cup R2)} (f_2 (f_1\ a))$ by the transitivity of
    $refltrans$.} \begin{coqdoccode}
\coqdocemptyline
\coqdocindent{3.00em}
\ensuremath{\times} \coqdoctac{apply} \coqdocvar{refltrans\_union}. \coqdoctac{apply} \coqdocvar{H2}. \end{coqdoccode}
\comm{From hypothesis $H2$, we
        know that $a \tto_{R1} (f_1\ a)$ for all $a$, and hence
        $a\tto_{(R1\cup R2)} (f_1\ a)$ and we conclude.} \begin{coqdoccode}
\coqdocemptyline
\coqdocindent{3.00em}
\ensuremath{\times} \coqdoctac{apply} \coqdocvar{H1} \coqdoctac{in} \coqdocvar{H}. \coqdoctac{rewrite} \coqdocvar{H}. \coqdoctac{apply} \coqdocvar{H3} \coqdockw{with} \coqdocvar{b}; \coqdoctac{reflexivity}. \end{coqdoccode}
        \comm{The proof that $(f_1\ b)\tto_{(R1\cup R2)} (f_2 (f_1\ a))$ is
        exactly the hypothesis $H3$.} \begin{coqdoccode}
\coqdocemptyline
\coqdocindent{2.00em}
+ \coqdoctac{apply} \coqdocvar{H1} \coqdoctac{in} \coqdocvar{H}. \coqdoctac{rewrite} \coqdocvar{H}. \coqdoctac{apply} \coqdocvar{refl}. \end{coqdoccode}
\comm{The proof that $(f_2
    (f_1\ a)) \tto_{(R1\cup R2)} (f_2 (f_1\ b))$ is done using the
    reflexivity of $refltrans$ because $(f_2 (f_1\ a)) = (f_2 (f_1\
    b))$ by hypothesis $H1$.} \begin{coqdoccode}
\coqdocemptyline
\coqdocindent{1.00em}
- \coqdoctac{apply} \coqdocvar{H4}; \coqdoctac{assumption}. \end{coqdoccode}
\comm{When $a \to_{R2} b$ then we are done by
    hypothesis $H4$.} \begin{coqdoccode}
\coqdocemptyline
\coqdocnoindent
\coqdockw{Qed}.\coqdoceol
\coqdocemptyline
\end{coqdoccode}

\flavio{ move to related work: In \cite{zproperty},
B. Felgenhauer et.al. formalized in Isabelle/HOL the Z property and
its relation to confluence.}

\section{Conclusion}

In this work we presented a constructive proof that the Z property
implies confluence, an important property for rewriting systems. In
addition, we formally proved this result in the Coq proof
assistant. The corresponding files are available in our GitHub
repository: \url{https://github.com/flaviodemoura/Zproperty}.

The Z property was presented by V. van Oostrom as a sufficient
condition for an ARS to be confluent~\cite{zproperty}, and since
then has been used to prove confluence in different contexts such as
the $\lambda$-calculus with $\beta\eta$-reduction, extensions of the
$\lambda$-calculus with explicit substitutions and the
$\lambda\mu$-calculus. The Coq proofs of the main results are
commented line by line which serve both as an informal presentation of
the proofs (i.e. proofs explained in natural language) and as its
formal counterpart. Moreover, we formalize an extension of the Z
property, known as compositional Z property, as presented in
\cite{Nakazawa-Fujita2016}

As future work, this formalization will be used to prove the
confluence property of a calculus with explicit substitution based on
the $\lambda_{ex}$-calculus (cf. ~\cite{kes09}). In addition, we hope
that our formalization can be used as a framework for proving
confluence of others rewriting systems.



\bibliographystyle{alpha} 
\bibliography{references}
%inline the .bbl file directly for mailing to authors.

% \begin{thebibliography}{Com79}

% \bibitem[Com79]{Comer-btree}
% D.~Comer.
% \newblock The ubiquitous b-tree.
% \newblock {\em Computing Surveys}, 11(2):121--137, June 1979.

% \bibitem[Knu73]{Knuth-vol3}
% D.~E. Knuth.
% \newblock {\em The Art of Computer Programming -- Volume 3 / Sorting and
%   Searching}.
% \newblock Addison-Wesley, 1973.

% \end{thebibliography}

\end{document}


\newpage

\flavio{\begin{center} As páginas seguintes ainda não foram revisadas \end{center}}
