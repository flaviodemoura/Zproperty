\documentclass[11pt]{article}

%use one of the following package accordingly

%\usepackage[brazil]{babel} % for portuguese
\usepackage[english]{babel} % for english
%\usepackage[spanish]{babel} % for spanish

\usepackage[latin1]{inputenc} % for accents in portuguese
%\usepackage[utf8]{inputenc} % for accents in portuguese using Unicode 
%%
%%
%% PLEASE DO NOT MAKE CHANGES TO THIS TEMPLATE 
%% THAT CAUSE CHANGES IN THE FORMAT OF THE TEXT
%%
%%
\usepackage[centertags]{amsmath}
\usepackage{indentfirst,amsfonts,amssymb,amsthm}
\usepackage{cite}
\usepackage{pdfsync}
\usepackage{url}
\usepackage[bottom=1.5cm,top=1.5cm,left=3cm,right=2cm]{geometry}
\date{}

\begin{document}

%********************************************************
\title{Confluence of an Explicit Substitutions Calculus Formalized}

\author{
  {\large Fl\'avio L. C. de Moura}\thanks{flaviomoura@unb.br},
    {\large Leandro O. Rezende}\thanks{L-ordo.ab.chao@hotmail.com}\\
  {\small Departamento de Ci\^encia da Computa\c{c}\~ao}\\
  {\small Universidadede Bras\'ilia, Bras\'ilia, Brazil} }

\maketitle

\begin{abstract}
%the abstract should contain a maximum of 600 words
  Rewriting theory is a well established model of computation
  equivalent to the Turing machines. The most well known rewriting
  system is the $\lambda$-calculus, the theoretical foundation of the
  functional paradigm of programming. Confluence is an important
  property related to the determinism of the results given by a
  rewriting system. In this work, which is still in progress, we
  formalize the confluence property of an extension of the
  $\lambda$-calculus with explicit substitutions following the steps
  in \cite{DK08,kes09}. Confluence is obtained through the Z
  property~\cite{zproperty}: we first formalized the fact that an
  abstract rewriting system, i.e. a binary relation over an arbitrary
  set, that satisfies the Z property is confluent. The formalization
  is done in the Coq proof assistant~\cite{CoqTeam}.

  In the $\lambda$-calculus, terms that differ by the name of its
  bound variables are considered equal. This notion is known as
  $\alpha$-equivalence, which is a costly computational equivalence
  relation. Alternatives to $\alpha$-equivalence include the so called
  De Bruijn indexes~\cite{dB72}, where variables are represented by
  natural numbers. In De Bruijn notation terms have a unique
  representation, and hence there is no need of
  $\alpha$-equivalence. Nevertheless, defining a reduction in De
  Bruijn notation requires a non-trivial algebra for referencing and
  updating indexes. The Locally Nameless Representation~\cite{Ch11} is
  a framework that takes the advantages of the two notations: bound
  variables are represented as De Bruijn indexes, while free variables
  uses names. The original framework uses classical logic and was
  built for representing $\lambda$-terms, therefore we decided to take
  some of its constructions (which are not based on classical logic)
  and define an operator for the substitution operation. Therefore,
  our framework is constructive, i.e. it does not rely on the law of
  excluded middle or on the proof by contradiction principles. This is
  important because one of the goals of this work is the generation of
  certified code via the extraction mechanism of Coq.

  Our formalization is based on the paper~\cite{DK08}, where the
  $\lambda$ex-calculus is defined. One of the challenging steps of
  this formalization is that the $\lambda$ex-calculus defines an
  equational theory based on the fact that reduction is done modulo
  permutation of independent substitutions. In order to avoid an
  explicit manipulation of permutation of independent substitutions,
  we use the generalized rewriting facilities of
  Coq~\cite{Sozeau09}. Nevertheless, the generated equivalence
  relation needs to be defined over every expression, and not only
  over terms. In order to circumvent this problem, we proved that the
  reduction relation defined by the $\lambda$ex-calculus modulo
  permutations of independent substitutions is restricted to terms.

  The formalization is available at
  \url{https://github.com/flaviodemoura/Zproperty.git} and is divided
  in two files:
  \begin{enumerate}
  \item The file {\tt ZtoConfl.v} contains the proof that an abstract
    rewriting system $R$ that satisfies the Z property is confluent;
  \item The file {\tt lex.v} contains the current status of the
    formalization showing that the $\lambda$ex-calculus in the locally
    nameless representation satisfies the Z-property, and hence is
    confluent.
  \end{enumerate}  
\end{abstract}

\bibliography{references}
 \bibliographystyle{plain}

\end{document}

%%% Local Variables: 
%%% mode: latex
%%% TeX-master: t
%%% End: 
