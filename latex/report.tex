\documentclass{llncs}

\usepackage{graphicx}
\usepackage{xspace}
\usepackage[utf8]{inputenc}
\usepackage{hyperref}
\usepackage{srcltx}
\usepackage{pdfsync}
\usepackage{fullpage}
\usepackage[all]{xy}
\usepackage{mathpartir}
\usepackage[english]{babel}
\usepackage[usenames]{color}
\usepackage{amsmath,amssymb}
\usepackage[color]{coqdoc}

\newcommand{\fv}[1]{{\tt fv}(#1)}

\newcommand{\flavio}[1]{{\color{red} #1}}
\newcommand{\leandro}[1]{{\color{blue} #1}}

\title{Confluence of Explicit Substitutions Calculi Formalized}
\author{Flávio L. C. de Moura and Leandro O. Rezende}
\institute{Departamento de Ciência da Computação, Universidade de Brasília}

\begin{document}
\maketitle

\begin{abstract}
  Rewriting theory is a well established model of computation
  equivalent to the Turing machines, and the most well known rewriting
  system is the $\lambda$-calculus, the theoretical foundation of the
  functional paradigm of programming. Confluence is an important
  property related to the determinism of the results given by a
  rewriting system. In this work, we formalize the confluence of an
  extension of the $\lambda$-calculus with explicit substitutions
  following the steps in \cite{kes09}. The formalization is done in
  the Coq proof assistant~\cite{CoqTeam}, and the whole proof is
  constructive, i.e. it does not rely on the law of excluded middle or
  on the proof by contradiction principles. This is important because
  the last step of this project aims the generate certified code via
  the extraction mechanism of Coq.
\end{abstract}

\section{Introduction}

This work is divided into two parts: The first part is about a
characterization of the confluence property for ARSs (Abstract
Rewriting Systems). An ARS is simply a pair composed of a set and
binary operation over this set. Given an ARS $(A,\to)$, where $A$ is a
set and $\to:A\times A$ is a binary relation over $A$, an expression
$a\in A$ usually can be reduced in different ways in order to produce
a result. Informally, the confluence property states that, no matter
how the reduction is done, the result will always be the same. The
following diagram is used to express this idea:

\[
  \xymatrix{
    & a \ar@{->>}[dl] \ar@{->>}[dr] & \\
    b \ar@{.>>}[dr] &  & c \ar@{.>>}[dl] \\
    & d & 
  }
\]

The diagram states that if the expression $a$ can be reduced in two
different ways to the expressions $b$ and $c$, then there exists an
expression $d$ such that both $b$ and $c$ reduces to $d$. In
~\cite{ZPropertyDraft}, a different characterization of confluence is
given by the so called $Z$-property:

\begin{definition}
      Let $(A,\to)$ be an abstract rewriting system (ARS). The system
    $(A,\to)$ has the Z property, if there exists a map $f:A \to A$
    such that:
    
    \[
      \xymatrix{
        a \ar[r] &  b \ar@{.>>}[dl]\\
        f(a) \ar@{.>>}[r] & f(b) \\ 
      }
    \]
\end{definition}

So the first part of this work was to formally prove that if an ARS $(A,\to)$ satisfies the $Z$-property then $(A,\to)$ is confluent. This part is complete. 

The second part of this work is to use the first one to get confluence of an extension of the $\lambda$-calculus with explicit substitutions known as $\lambda$ex-calculus~\cite{kes09}.

\subsection{Explicit Substitutions Calculi}

\flavio{Inserir um pequeno histórico sobre cálculos com substituições explícitas incluindo as propriedades importantes para estes cálculos: confluência (em termos fechados e abertos), simulação de um passo de $\beta$-conversão, terminação do cálculo de substituições associado e preservação da normalização forte.}

\subsubsection{The $\lambda$ex-calculus }


Calculi with explicit substitutions are extensions of the
$\lambda$-calculus in which the substitution operation is a primitive
operation, making them a formalism that is closer to implementations
based on the $\lambda$-calculus. During the last three decades, this
area of research attracted the attention of the scientific community
~\cite{Lins86,accl91,Bloo95,BeBrLeRD96,jfp,Munoz96,NaWi98,kes09}. The
main reason for the development of several different calculi with
explicit substitutions was that none of them were faithful to the
system they were supposed to model: the $\lambda$-calculus. This means
that none of these calculi satisfy simultaneously a set of important
properties: Confluence on open and closed terms, Simulation of one
step $\beta$-reduction, Termination of the associated substitution
calculus and Preservation of Strong Normalisation. In 2009, Delia
Kesner published a paper in which she presents a calculus that
satisfies all these properties: the $\lambda$ex-calculus. The terms of
the $\lambda$ex-calculus is given by the following grammar, equation and rules:

$$\mbox{\tt (terms)}\;\;\;\;\;\;\; t ::= x \mid t\ t \mid \lambda x.t \mid t[x/t]$$

$$\begin{array}{llll}
    t[x/u][y/v] & =_C & t[y/v][x/u] & \mbox{if } y\notin \fv{u} \\
                &&& \mbox{and } x\notin \fv{v} \\
    (\lambda x.t) u & \to_{\tt B} & t[x/u] & \\
    x[x/u] & \to_{\tt Var} & u & \\      
    t[x/u] & \to_{\tt Gc} & t, & \mbox{ if } x \notin\fv{t} \\      
    (t\ v)[x/u] & \to_{\tt App} & t[x/u]\ v[x/u] & \\
    (\lambda y. t)[x/u] & \to_{\tt Lamb} & \lambda y.t[x/u] & \\
    t[x/u][y/v] & \to_{\tt Comp} & t[y/v][x/u[y/v]], & \mbox{ if } y \in\fv{u}            
  \end{array}$$
  
  \subsection{The Framework}

  \flavio{Detalhar}

  A direct implementation or formalization of the $\lambda$ex-calculus
  is not straightforward because terms are considered modulo renaming
  of bound variables, i.e. modulo $\alpha$-conversion. A interesting
  way to avoid to deal with $\alpha$-conversion is to use the so
  called {it De Bruijn}
  notation~\cite{bruijn72:_lambd_churc_rosser}. Nevertheless, De
  Briuijn notation also has it disadvantages: the manipulation of free
  variables is tricky. The chosen framework for our formalization
  takes the benefits of named and De Bruijn notation. It is known as
  Locally Nameless Representation ~\cite{Ch11}: it uses names for free
  variables and De Bruijn indexes for bound variables.

\section{The Formalization}
  
Our formalization is being done in Coq~\cite{CoqTeam}, a constructive
proof assistant written in the OCaml programming language. The
$Z$ and confluence properties are written in the Coq language as follows:

\begin{coqdoccode}
  \coqdocnoindent \coqdockw{Definition}
  \coqdocvar{Zprop} \{\coqdocvar{A}:\coqdockw{Type}\}
  (\coqdocvar{R}: \coqdocvar{Rel} \coqdocvar{A}) :=
  \coqdoctac{\ensuremath{\exists}} \coqdocvar{f}:\coqdocvar{A}
  \ensuremath{\rightarrow} \coqdocvar{A},
  \coqdockw{\ensuremath{\forall}} \coqdocvar{a} \coqdocvar{b},
  \coqdocvar{R} \coqdocvar{a} \coqdocvar{b} \ensuremath{\rightarrow}
  ((\coqdocvar{refltrans} \coqdocvar{R}) \coqdocvar{b}
  (\coqdocvar{f} \coqdocvar{a}) \ensuremath{\land}
  (\coqdocvar{refltrans} \coqdocvar{R}) (\coqdocvar{f}
  \coqdocvar{a}) (\coqdocvar{f} \coqdocvar{b})).\coqdoceol
\end{coqdoccode}

\begin{coqdoccode}
  \coqdocnoindent\coqdockw{Definition} \coqdocvar{Confl}
  \{\coqdocvar{A}:\coqdockw{Type}\} (\coqdocvar{R}: \coqdocvar{Rel}
  \coqdocvar{A}) := \coqdockw{\ensuremath{\forall}} \coqdocvar{a}
  \coqdocvar{b} \coqdocvar{c}, (\coqdocvar{refltrans} \coqdocvar{R})
  \coqdocvar{a} \coqdocvar{b} \ensuremath{\rightarrow}
  (\coqdocvar{refltrans} \coqdocvar{R}) \coqdocvar{a} \coqdocvar{c}
  \ensuremath{\rightarrow} (\coqdoctac{\ensuremath{\exists}}
  \coqdocvar{d}, (\coqdocvar{refltrans} \coqdocvar{R}) \coqdocvar{b}
  \coqdocvar{d} \ensuremath{\land} (\coqdocvar{refltrans}
  \coqdocvar{R}) \coqdocvar{c} \coqdocvar{d}).\coqdoceol
\end{coqdoccode}

The first part of this work aimed to prove the following theorem:

\begin{coqdoccode}
\coqdocnoindent
\coqdockw{Theorem} \coqdocvar{Zprop\_implies\_Confl} \{\coqdocvar{A}:\coqdockw{Type}\}: \coqdockw{\ensuremath{\forall}} \coqdocvar{R}: \coqdocvar{Rel} \coqdocvar{A}, \coqdocvar{Zprop} \coqdocvar{R} \ensuremath{\rightarrow} \coqdocvar{Confl} \coqdocvar{R}.\coqdoceol
\end{coqdoccode}
  
The challenging point of this proof was the structure of the induction on the number of steps of \coqdocvar{refltrans} \coqdocvar{R} \coqdocvar{a} \coqdocvar{b} and \coqdocvar{refltrans} \coqdocvar{R} \coqdocvar{a} \coqdocvar{c}. Since there is no information of the number of steps of either \coqdocvar{refltrans} \coqdocvar{R} \coqdocvar{a} \coqdocvar{b} or \coqdocvar{refltrans} \coqdocvar{R} \coqdocvar{a} \coqdocvar{c}, just doing induction on one followed by the other is not enough, as it would require us to prove that \coqdocvar{refltrans} \coqdocvar{R} \coqdocvar{b} \coqdocvar{c} or \coqdocvar{refltrans} \coqdocvar{R} \coqdocvar{c} \coqdocvar{b}. Luckily, we can use induction on \coqdocvar{refltrans} \coqdocvar{R} \coqdocvar{a} \coqdocvar{b}, then discard the information of the reduction from \coqdocvar{a} to \coqdocvar{b'} (where \coqdocvar{R} \coqdocvar{a} \coqdocvar{b'} and \coqdocvar{refltrans} \coqdocvar{R} \coqdocvar{b'} \coqdocvar{b}) after we use the $Z$-property to assure that \coqdocvar{refltrans} \coqdocvar{R} \coqdocvar{a} (\coqdocvar{f} \coqdocvar{a}) \ensuremath{\land} \coqdocvar{refltrans} \coqdocvar{R} \coqdocvar{b'} (\coqdocvar{f} \coqdocvar{a}), where \coqdocvar{f} \coqdocvar{a} comes from applying the $Z$-property. Then, our proof can be shown as the following diagram:

\[
  \xymatrix{
    & a \ar@{->>}[d] \ar@{->>}[rdd] & \\
    b' \ar@{->>}[r] \ar @{->>}[d] & f(a) &  \\
    b \ar @{.>>}[dr] & & c \ar@{.>>}[ld] \\
    & d &
  }
\]

Now, we use induction on \coqdocvar{refltrans} \coqdocvar{R} \coqdocvar{a} \coqdocvar{c}, use the fact that \coqdocvar{refltrans} \coqdocvar{R} \coqdocvar{a} (\coqdocvar{f} \coqdocvar{a}), coupled with the induction hypothesis from the induction on \coqdocvar{refltrans} \coqdocvar{R} \coqdocvar{a} \coqdocvar{b}, to close the induction basis.

\[
  \xymatrix{
    & a \ar@{->>}[d] \ar@{->>}[rd] \\
    b' \ar@{->>}[r] \ar @{->>}[d] & f(a) & c' \ar [d] \\
    b \ar @{.>>}[dr] & & c \ar@{.>>}[ld] \\
    & d &
  }
\]

Now, to close the proof, we can use the two induction hypotheses to conclude the proof that the $Z$-property implies confluence.

According to the locally nameless representation variables must be split in two classes because the bound ones are coded as indexes, while the free ones use names:

  \begin{coqdoccode}
    \coqdocnoindent \coqdockw{Inductive} \coqdocvar{pterm} :
    \coqdockw{Set} :=\coqdoceol \coqdocindent{1.00em} \ensuremath{|}
    \coqdocvar{pterm\_bvar} : \coqdocvar{nat} \ensuremath{\rightarrow}
    \coqdocvar{pterm}\coqdoceol \coqdocindent{1.00em} \ensuremath{|}
    \coqdocvar{pterm\_fvar} : \coqdocvar{var} \ensuremath{\rightarrow}
    \coqdocvar{pterm}\coqdoceol \coqdocindent{1.00em} \ensuremath{|}
    \coqdocvar{pterm\_app} : \coqdocvar{pterm}
    \ensuremath{\rightarrow} \coqdocvar{pterm}
    \ensuremath{\rightarrow} \coqdocvar{pterm}\coqdoceol
    \coqdocindent{1.00em} \ensuremath{|} \coqdocvar{pterm\_abs} :
    \coqdocvar{pterm} \ensuremath{\rightarrow}
    \coqdocvar{pterm}\coqdoceol \coqdocindent{1.00em} \ensuremath{|}
    \coqdocvar{pterm\_sub} : \coqdocvar{pterm}
    \ensuremath{\rightarrow} \coqdocvar{pterm}
    \ensuremath{\rightarrow} \coqdocvar{pterm}.\coqdoceol
  \end{coqdoccode}

  This new grammar requires some caution because there are some expressions generated by it that do not correspond to a term. In fact, a expression formed by a sole De Bruijn index is not a term because indexes are intended to represent bound variables only. Therefore, the expressions that correspond to the usual notion of term is a proper subset of the above grammar and is characterized by the following predicate:

  \begin{coqdoccode}
\coqdocemptyline
\coqdocnoindent
\coqdockw{Inductive} \coqdocvar{term} : \coqdocvar{pterm} \ensuremath{\rightarrow} \coqdockw{Prop} :=\coqdoceol
\coqdocindent{1.00em}
\ensuremath{|} \coqdocvar{term\_var} : \coqdockw{\ensuremath{\forall}} \coqdocvar{x},\coqdoceol
\coqdocindent{3.00em}
\coqdocvar{term} (\coqdocvar{pterm\_fvar} \coqdocvar{x})\coqdoceol
\coqdocindent{1.00em}
\ensuremath{|} \coqdocvar{term\_app} : \coqdockw{\ensuremath{\forall}} \coqdocvar{t1} \coqdocvar{t2},\coqdoceol
\coqdocindent{3.00em}
\coqdocvar{term} \coqdocvar{t1} \ensuremath{\rightarrow} \coqdoceol
\coqdocindent{3.00em}
\coqdocvar{term} \coqdocvar{t2} \ensuremath{\rightarrow} \coqdoceol
\coqdocindent{3.00em}
\coqdocvar{term} (\coqdocvar{pterm\_app} \coqdocvar{t1} \coqdocvar{t2})\coqdoceol
\coqdocindent{1.00em}
\ensuremath{|} \coqdocvar{term\_abs} : \coqdockw{\ensuremath{\forall}} \coqdocvar{L} \coqdocvar{t1},\coqdoceol
\coqdocindent{3.00em}
(\coqdockw{\ensuremath{\forall}} \coqdocvar{x}, \coqdocvar{x} \symbol{92}\coqdocvar{notin} \coqdocvar{L} \ensuremath{\rightarrow} \coqdocvar{term} (\coqdocvar{t1} \^{} \coqdocvar{x})) \ensuremath{\rightarrow}\coqdoceol
\coqdocindent{3.00em}
\coqdocvar{term} (\coqdocvar{pterm\_abs} \coqdocvar{t1})\coqdoceol
\coqdocindent{1.00em}
\ensuremath{|} \coqdocvar{term\_sub} : \coqdockw{\ensuremath{\forall}} \coqdocvar{L} \coqdocvar{t1} \coqdocvar{t2},\coqdoceol
\coqdocindent{2.50em}
(\coqdockw{\ensuremath{\forall}} \coqdocvar{x}, \coqdocvar{x} \symbol{92}\coqdocvar{notin} \coqdocvar{L} \ensuremath{\rightarrow} \coqdocvar{term} (\coqdocvar{t1} \^{} \coqdocvar{x})) \ensuremath{\rightarrow}\coqdoceol
\coqdocindent{3.00em}
\coqdocvar{term} \coqdocvar{t2} \ensuremath{\rightarrow} \coqdoceol
\coqdocindent{3.00em}
\coqdocvar{term} (\coqdocvar{pterm\_sub} \coqdocvar{t1} \coqdocvar{t2}).\coqdoceol
\end{coqdoccode}

In order to prove that the $\lambda$ex-calculus has the $Z$-property, one needs to find a function $f$ from terms to terms that satisfies the definition. Following the steps of ~\cite{kes09} we take $f$ as the superdevelopment function:

\begin{coqdoccode}
\coqdocemptyline
\coqdocnoindent
\coqdockw{Fixpoint} \coqdocvar{sd} (\coqdocvar{t} : \coqdocvar{pterm}) : \coqdocvar{pterm} :=\coqdoceol
\coqdocindent{1.00em}
\coqdockw{match} \coqdocvar{t} \coqdockw{with}\coqdoceol
\coqdocindent{1.00em}
\ensuremath{|} \coqdocvar{pterm\_bvar} \coqdocvar{i}    \ensuremath{\Rightarrow} \coqdocvar{t}\coqdoceol
\coqdocindent{1.00em}
\ensuremath{|} \coqdocvar{pterm\_fvar} \coqdocvar{x}    \ensuremath{\Rightarrow} \coqdocvar{t}\coqdoceol
\coqdocindent{1.00em}
\ensuremath{|} \coqdocvar{pterm\_abs} \coqdocvar{t1}    \ensuremath{\Rightarrow} \coqdocvar{pterm\_abs} (\coqdocvar{sd} \coqdocvar{t1})\coqdoceol
\coqdocindent{1.00em}
\ensuremath{|} \coqdocvar{pterm\_app} \coqdocvar{t1} \coqdocvar{t2} \ensuremath{\Rightarrow} \coqdockw{let} \coqdocvar{t0} := (\coqdocvar{sd} \coqdocvar{t1}) \coqdoctac{in}\coqdoceol
\coqdocindent{11.00em}
\coqdockw{match} \coqdocvar{t0} \coqdockw{with}\coqdoceol
\coqdocindent{11.00em}
\ensuremath{|} \coqdocvar{pterm\_abs} \coqdocvar{t'} \ensuremath{\Rightarrow} \coqdocvar{t'} \^{}\^{} (\coqdocvar{sd} \coqdocvar{t2})\coqdoceol
\coqdocindent{11.00em}
\ensuremath{|} \coqdocvar{\_} \ensuremath{\Rightarrow} \coqdocvar{pterm\_app} (\coqdocvar{sd} \coqdocvar{t1}) (\coqdocvar{sd} \coqdocvar{t2})\coqdoceol
\coqdocindent{11.00em}
\coqdockw{end} \coqdoceol
\coqdocindent{1.00em}
\ensuremath{|} \coqdocvar{pterm\_sub} \coqdocvar{t1} \coqdocvar{t2} \ensuremath{\Rightarrow} (\coqdocvar{sd} \coqdocvar{t1}) \^{}\^{} (\coqdocvar{sd} \coqdocvar{t2})\coqdoceol
\coqdocindent{1.00em}
\coqdockw{end}.\coqdoceol
\end{coqdoccode}

\noindent where \begin{coqdoccode}\coqdocvar{t} \^{}\^{} \coqdocvar{t'}\end{coqdoccode} corresponds to the term obtained from  \begin{coqdoccode}\coqdocvar{t}\end{coqdoccode} after replacing all occurrences of its first dangling De Bruijn index by  \begin{coqdoccode}\coqdocvar{t'}\end{coqdoccode}, i.e. the operation  \begin{coqdoccode}\^{}\^{}\end{coqdoccode} simulates the metasubstitution of the $\lambda$-calculus. The following challenging step is to prove that this function allows us to prove that the  $\lambda$ex-calculus satisfies the $Z$-property, and then conclude that it is confluent:

\begin{coqdoccode}
\coqdocnoindent
\coqdockw{Lemma} \coqdocvar{BxZlex}: \coqdockw{\ensuremath{\forall}} \coqdocvar{a} \coqdocvar{b}, \coqdocvar{a} \ensuremath{\rightarrow}\coqdocvar{\_lex} \coqdocvar{b} \ensuremath{\rightarrow} \coqdocvar{b} \ensuremath{\rightarrow}\coqdocvar{\_lex}\ensuremath{\times} (\coqdocvar{sd} \coqdocvar{a}) \ensuremath{\land} (\coqdocvar{sd} \coqdocvar{a}) \ensuremath{\rightarrow}\coqdocvar{\_lex}\ensuremath{\times} (\coqdocvar{sd} \coqdocvar{b}).\coqdoceol
\coqdocemptyline
\coqdocnoindent
\coqdockw{Theorem} \coqdocvar{Zlex}: \coqdocvar{Zprop} \coqdocvar{lex}.\coqdoceol
\coqdocnoindent
\coqdockw{Proof}.\coqdoceol
\coqdocindent{1.00em}
\coqdoctac{unfold} \coqdocvar{Zprop}.\coqdoceol
\coqdocindent{1.00em}
\coqdoctac{\ensuremath{\exists}} \coqdocvar{sd}.\coqdoceol
\coqdocindent{1.00em}
\coqdoctac{apply} \coqdocvar{BxZlex}.\coqdoceol
\coqdocnoindent
\coqdockw{Qed}.\coqdoceol
\coqdocemptyline
\coqdocnoindent
\coqdockw{Corollary} \coqdocvar{lex\_is\_confluent}: \coqdocvar{Confl} \coqdocvar{lex}.\coqdoceol
\coqdocnoindent
\coqdockw{Proof}.\coqdoceol
\coqdocindent{1.00em}
\coqdoctac{apply} \coqdocvar{Zprop\_implies\_Confl}.\coqdoceol
\coqdocindent{1.00em}
\coqdoctac{apply} \coqdocvar{Zlex}.\coqdoceol
\coqdocnoindent
\coqdockw{Qed}.\coqdoceol
\end{coqdoccode}

\section{Conclusion}

By concluding this work, we will have formalized the confluence
property of an extension of the $\lambda$-calculus. A parallel project
is developing the formalization of the termination property of the
$\lambda$ex-calculus. By merging both formalizations we aim to
automatically extract certified Ocaml code of an explicit
substitutions calculus that satisfies all the desired properties:
confluence, simulation of one-step $\beta$-reduction, termination of
the associated calculus and preservation of strong normalization.


\bibliographystyle{plain}
\bibliography{../../../Dropbox/bibliography/references}

\end{document}

%%% Local Variables: 
%%% mode: latex
%%% TeX-master: t
%%% End: 